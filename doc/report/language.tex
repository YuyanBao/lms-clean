\section{Language}
\label{sec:language}

\subsection{Modifiers}
A type is either a type variable or consists of an modifier, a type name, and possibly type arguments. There are three types of modifiers.
\begin{itemize}
\item \emph{local} expresses that an object does not escape from its defining method.
\item \emph{sep} expresses that the memory regions reachable from an object is disjoint with other objects.
\item \emph{transient} in \cite{haller2010capabilities} and \emph{references and borrowing} (\KW{mut} and \KW{shr}) in Rust expresses that the passing object's permission will be returned back to the caller once the method finishes executing. In another words, the ownership is \emph{``transfered" with limited lifetime}. (\KW{mut} expresses that the object is mutable. \KW{shr} expreses that the object is read-only.)

Without the \emph{transient} or \KW{mut} or {shr} modifier, the caller loses the permission of the object. (Ownership transfer (Separation logic, Rust))
%\begin{lstlisting}[language=Scala,basicstyle=\footnotesize\ttfamily]
%let s = String::from("hello")
%takes_ownership(s);      // s lost the ownership
%
%let s1 = gives_ownership();  // s1 got the ownership
%
%fn takes_ownership(some_string: String) {
%   println!("{}", some_string);
%}
%
%fn gives_ownership() -> String {
%  let some_string = String :: from("hello");
%  some_string
%}
%\end{lstlisting}

\end{itemize}

\begin{figure}[!htb]
\small
\centering
\begin{grammar}%
\begin{tabular}{r r l}
%$~$\nonterm{md} & \: & \nonterm{m}(\nonterm{x}: \nonterm{T}):\nonterm{T} \{ \nonterm{t} \}
$~$\nonterm{t} & \: & \nonterm{c} \| \nonterm{x} \| \KW{new} $\mu_1\mu_2$\nonterm{T} \| $\lambda$\nonterm{x}:$\mu_1\mu_2$\nonterm{T}.\nonterm{t} \| \nonterm{t}\nonterm{t} \| $\lambda$\nonterm{X}.\nonterm{t} \| \nonterm{t}[\nonterm{T}] 
& & \| \{\nonterm{t},\nonterm{t}\} \| \nonterm{t}.$1$ \| \nonterm{t}.$2$ 
$~$\nonterm{v} & \: & \nonterm{c} \| $\lambda$\nonterm{x}:$\mu_1\mu_2$\nonterm{T}.\nonterm{t} \| $\lambda$\nonterm{X}.\nonterm{t} \| \{\nonterm{v},\nonterm{v}\}
$~$\nonterm{T}$_s$ & \: & \nonterm{B} \| \nonterm{X} \| \nonterm{T}$_s\to$\nonterm{T}$_s$ \| $\forall$\nonterm{X}.\nonterm{T}$_s$ \| \nonterm{T}$$_s\times$\nonterm{T}$_s$
$~\Gamma_s$ & \: & $\emptyset$ \| $\Gamma$, \nonterm{x}:\nonterm{T}$_s$ \| $\Gamma$,\nonterm{X}
$~\mu_1$ & \: & \KW{any} \| \KW{sep}
$~\mu_2$ & \: & \KW{any} \| \KW{local} \| \KW{mut} \| \KW{shr}
\end{tabular} %
\end{grammar}
\caption{The syntax of the programming language.}
\label{fig:syntax}
\end{figure}
\subsection{Types}
\begin{figure}[!htb]
\begin{mathpar}
\inferrule[(\emph{O\_Intro})]{ }{mut \MSUBTYPE shr}

\inferrule[(\emph{O\_Ma})]{ }{\mu \MSUBTYPE any}

\inferrule[(\emph{O\_Refl})]{ }{\mu \MSUBTYPE \mu}
\end{mathpar}
\caption{Ordering of modifier}
\label{fig:order-mod}
\end{figure}
\begin{figure}[!htb]
\begin{mathpar}
\inferrule[(\emph{ST\_Refl})]{ }
{ \TypJudg{}{T ~ \SUBTYPE T}} 

\inferrule[(\emph{ST\_Fun})]{ \TypJudg{}{T_1 \SUBTYPE T_1'} \\ \TypJudg{}{T_2' \SUBTYPE T_2}}
{\TypJudg{}{T_1' \mapsto T_2' \SUBTYPE T_1 \mapsto T_2}}

\inferrule[(\emph{ST\_Trans})]{  \TypJudg{}{T_1 \SUBTYPE T_2} \\ \TypJudg{}{T_2 \SUBTYPE T_3 }}
{ \TypJudg{}{T_1 ~ \SUBTYPE T_3}} \\

\end{mathpar}
\caption{Static typing rules for subtypes}
\label{fig:subtypes}
\end{figure}

\subsection{Semantics}
Each closure instance will be identified by a unique token that represents the closure occurence from which it was created \cite{kassios2010specification} and method declarations. Tokens and memory addresses are disjoint. Let $\theta$ map from tokens to closure occurences or method declarations.

There is a store $s$ that maps from variables to values, and a heap $h$ that maps from locations to values. A location is a pair of an address and its owner. Addresses are natural numbers. 
Heap looks up is denoted by $h(l)$, where $l$ is a location. Heap extension is denoted by $h + (l, v)$, where $\RESTRICT{l}{1} \notin \RESTRICT{\DOM(h)}{1}$. 
Ownership updates is denoted by $h[addr, o]$, where $(addr, \_) \in \DOM(h)$, and $for all (addr', o') \in \DOM(h). (addr' \neq addr \Rightarrow h[addr, o](addr', o') = h(addr', o')) \land addr == addr' \Rightarrow h[addr, o](addr, o) = h(addr', o')$.

\begin{figure}[!ht]
\small
\centering
\begin{grammar}%
\begin{tabular}{r r l}
$~$\nonterm{addr} & \: & $\mathbar{N}$
$~$\nonterm{l} & \: & (\nonterm{addr},\nonterm{o})
$~$\nonterm{s} & \: & $\overline{x \mapsto v}$
$~$\nonterm{h} & \: & $\overline{l \mapsto v}$
$~$\nonterm{o} & \: & $\epsilon$ \| \nonterm{addr} \| \nonterm{id} 
$~$\nonterm{v} & \: & \nonterm{l} \| \nonterm{null}$_l$ \| $\lambda$\nonterm{x}.\nonterm{t} \| \{\nonterm{v},\nonterm{v}\}
$~\mu$ & \: & \KW{mut}\& \| \KW{shr}\& \| \KW{sep}
%$~\Gamma_r$ & \: & $\emptyset$ \| $\Gamma$, \nonterm{X} $\mapsto$ \nonterm{T}$_r$ \| $\Gamma$,\nonterm{x} $\mapsto$ v
\end{tabular} %
\end{grammar}
\caption{Definition of the runtime model.}
\label{fig:runtime}
\end{figure}


\figref{fig:semantics} shows the operational semantics. The computation tracks ownership.

\begin{figure}
\begin{mathpar}
\inferrule[(\emph{evar})]{s(x) = v }
{\MEVAL{(s, h)}{x}{(s, h), v}}

\inferrule[(\emph{enew})]{\KW{fresh}(addr) \\ l = (addr, id) \\ v = \DEFAULT(T, addr) \\ h' = h + (l, v)}
{\MEVAL{(s, h)}{\KW{new} ~ T}{(s, h'), l}}

%\inferrule[(\emph{eabs})]{ v = \lambda x . t \\ c' = c + (id, v) }
%{\MEVAL{(s, h)}{\lambda x:T_r.t}{(s, h'), v}}

\begin{array}{l}
\inferrule[(\emph{eproj})] { \MEVAL{(s, h)}{t}{l} \\ v = h(l)[i]}
{\MEVAL{(s, h)}{t.i}{(s, h), v}} \\
\end{array}

\inferrule[(\emph{eapp})]{ \theta(id_1) = \lambda x. t_3 \\ \MEVAL{(s, h)}{t_2}{(s, h), (addr, o)} \\ h_2 = (h_1[addr, \epsilon][addr_1, o_1]) \\ \NMEVAL{(s, x:(addr, o), h)}{t_3}{(s_1, h_1), (addr_1, o_1)}{id_1}}
{\MEVAL{(s, h)}{t_1 t_2}{(s, h_2), (addr_1, o_1)}}

\end{mathpar}
\caption{Operational Semantics}
\label{fig:semantics}
\end{figure}

\begin{figure}[!htb]
\begin{mathpar}
\inferrule[(\emph{tbnew})]{ }
{ \TypJudg{\Gamma}{\KW{new} ~ T : \KW{own} ~ T}} 

\inferrule[(\emph{tpair})]{\TypJudg{\Gamma}{t_1 : \KW{own} ~ T_1 \triangleright \epsilon}  \\ \TypJudg{\Gamma}{t_2: \KW{own} ~ T_2 \triangleright \epsilon}} 
{ \TypJudg{\Gamma}{\{t_1, t_2\} : \KW{own} ~ T_1 \times T_2} }

\inferrule[(\emph{tproj})]{\TypJudg{\Gamma}{t : \KW{own} ~ T_1  \times T_2 \triangleright \epsilon;} \\ \Gamma' = \Gamma + (t.1:\KW{own} ~ T_1) + (t.2 : \KW{own} ~ T_2)}
{ \TypJudg{\Gamma'}{t.1 : \KW{own} ~ T_1 } }


\inferrule[(\emph{tbmute})]{ x : \KW{own} ~ T \in \Gamma }
{ \TypJudg{\Gamma}{\KW{mut} \& ~ x : \REF_{mut}}} 

\inferrule[(\emph{tbreborrow})]{ x : \REF_{mut} ~ T \in \Gamma }
{ \TypJudg{\Gamma}{\KW{mut} \& ~ x : \REF_{mut} ~ T}} 

\inferrule[(\emph{tbshare})]{ x : \KW{own} ~ T \in \Gamma }
{ \TypJudg{\Gamma}{\KW{shr} \& ~ x : \REF_{shr} ~ T}}


\end{mathpar}
\caption{Static typing rules for ownership and borrowing}
\label{fig:subtypes}
\end{figure}

\begin{figure}[!htb]
\begin{mathpar}
\inferrule[(\emph{tsnew})]{ }
{ \TypJudg{\Gamma}{\KW{new} ~ T : \KW{sep}, \KW{own} ~ T}} 

\inferrule[(\emph{tspair})]{\TypJudg{\Gamma}{t_1 : \KW{sep}, \KW{own} ~ T_1 \triangleright  \epsilon } \\ \TypJudg{\Gamma}{t_2 : \KW{sep}, \KW{own} ~ T_2 \triangleright  \epsilon} }
{\TypJudg{\Gamma}{ \{t_1, t_2\} : \KW{sep}, \KW{own} ~ T_1 \times T_2}}

\end{mathpar}
\caption{Static typing rules for separation and ownership}
\label{fig:subtypes}
\end{figure}