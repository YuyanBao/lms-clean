\usepackage{listings,array,varwidth}
\usepackage{mathpartir}
\usepackage{mathtools}

\usepackage{diagbox}
\usepackage{float}
\usepackage{varioref}
\usepackage{xfrac}
\usepackage{xcolor}
\renewcommand{\ttdefault}{pcr} % an alternative for bold tt fonts, if luximono doesn't work
\usepackage{enumitem}
\usepackage[skip=2.5pt]{caption}
\usepackage{slashbox}


\makeatletter % allow us to mention @-commands
\def\arcr{\@arraycr}
\makeatother

%%----------------------------
% macro to try to fix the bibliography problems, but this doesn't work...
%\def\showDOI#1{\mbox{{\tt DOI:}{#1}}}
\newcommand{\showDOI}[1]{\unskip}
%%----------------------------

\newcommand{\tikzmark}[1]{\tikz[overlay,remember picture] \node (#1) {};}

\newtheorem{innercustomgeneric}{\customgenericname}
\providecommand{\customgenericname}{}
\newcommand{\newcustomtheorem}[2]{%
  \newenvironment{#1}[1]
  {%
   \renewcommand\customgenericname{#2}%
   \renewcommand\theinnercustomgeneric{##1}%
   \innercustomgeneric
  }
  {\endinnercustomgeneric}
}
\newcustomtheorem{re-definition}{Definition}
\newcustomtheorem{re-lemma}{Lemma}


\newcommand*{\Scale}[2][4]{\scalebox{#1}{$#2$}}%
\newcommand{\nonterm}[1]{\textrm{\textit{#1}}}

\newcommand{\figref}[1]{Fig.~\ref{#1}}  
\newcommand{\secref}[1]{Section~\ref{#1}} % uses varioref
\newcommand{\thmref}[1]{Theorem~\ref{#1}}
\newcommand{\lemref}[1]{Lemma~\ref{#1}}
\newcommand{\colref}[1]{Corollary~\ref{#1}}
\newcommand{\defref}[1]{Def.~\ref{#1}}
\newcommand{\appref}[1]{Appendix~\ref{#1}}
\newcommand{\tbref}[1]{Table~\ref{#1}}

% Theorems
%\newcommand{\PROOF}{\topsep 0pt \partopsep 0pt{\em Proof:\/}~~}
\newcommand{\QED}{\rule{0.4em}{0.65em}}

\newcommand{\Specsharp}{%
	{\settoheight{\dimen0}{C}Spec\kern-.05em \resizebox{!}{\dimen0}{\raisebox{\depth}{\#}}}}
\newcommand{\Csharp}{%
	{\settoheight{\dimen0}{C}C\kern-.05em \resizebox{!}{\dimen0}{\raisebox{\depth}{\#}}}}
	

\newcommand{\REGION}[1]{\ensuremath{\KW{region}\{#1\}}}
\newcommand{\THIS}{\KW{this}}
\newcommand{\REFINES}{\ensuremath{\sqsupseteq}}	
\newcommand{\REFINESAT}[1]{\ensuremath{\sqsupseteq}^{#1}}
\newcommand{\ATREFINES}[1]{\ensuremath{\sqsupseteq^{#1}}}
\newcommand{\DREFINES}[1]{\ensuremath{\sqsupseteq^{*#1}}}
\newcommand{\TRANS}{\ensuremath{\rightsquigarrow}}
\newcommand{\PROJ}[2]{\ensuremath{{#1}|_{#2}}}

% Separation Logic
\newcommand{\WAND}{\ensuremath{-\!\!*}}
\newcommand{\POINTSTO}{\mapsto}
\newcommand{\RESTRICT}[2]{\ensuremath{#1 \!\!\downharpoonright\!\! #2}}
\newcommand{\DRESTRICT}[2]{\ensuremath{#1 \!\!\downharpoonright^{*}\!\! #2}}
\newcommand{\UPCAST}[3]{\ensuremath{#1 \!\!\upharpoonright\!\!_{#2}^{#3}}}
\newcommand{\DRAND}{\ensuremath{\&\&}}
\newcommand{\DROR}{\ensuremath{||}}
\newcommand{\scon}{\mathbin{\varstar}}
\newcommand{\ocon}{%
  \mathbin{\mbox{$\mathrlap{\cup}\hspace*{.15em}
      \raisebox{.01em}[0ex][0ex]{$\scon$}$\hspace*{.07em}}}}
\newcommand{\LIMP}{\ensuremath{-\!\!\circ}}
      
% Region Logic      
\newcommand{\Img}{\mbox{\large\textbf{`}}}
\newcommand{\Sep}{{\bf \sfrac{\cdot}{\cdot}}}
\newcommand{\DInc}{\hspace{1em}\substack{\mathrlap{<} \\ \mathrlap {>}}\hspace{1em}}
\newcommand{\Immune}[2]{\ensuremath{{#1}/{#2}\mbox{-immune}}}

\newcommand{\UNDEF}{\bot}
\newcommand{\ERR}{\textit{err}}

% semantics
\newcommand{\VAL}{\textit{Val}}
      
% % Expression evaluation 
\newcommand{\MEANINGFUN}[1]{\mathcal{#1}}
\newcommand{\MEREG}[2]{\MEANINGFUN{R}\synbracket{\Gamma \vdash {#1} }{(#2)}}

\newcommand{\MO}[1]{{\mathcal{MO}} \synbracket{#1}}
\newcommand{\TYPEJUDG}[2] {{\Delta} \vdash^{\Gamma} {#1}: {#2}}
\newcommand{\STMTTYPEJUDG}[2] {{\Delta} \vdash^{\Gamma} {#1} \triangleright {#2}}

\newcommand{\BODY}[2]{\textit{body}( {#1}, {#2} )}
\newcommand{\SUBTYPE}{\ensuremath{\leq:}}
\newcommand{\OSUBTYPE}{\ensuremath{\leq_T}}
\newcommand{\ISREF}[1]{\textit{isRef}( {#1}  )  }

% domains and global variables
\newcommand{\DOMAIN}[1]{\fun{#1}}
\newcommand{\Heap}{\textit{Heap}}
\newcommand{\State}{\textit{State}}
\newcommand{\Store}{\DOMAIN{Store}}
\newcommand{\RFCTX}{\textit{RefCtx}}
\newcommand{\MEnv}{\textit{MethEnv}}
\newcommand{\ReferenceContext}{\DOMAIN{Reference-Context}}
\newcommand{\Loc}{\set{Loc}}
\newcommand{\Obj}{\textit{Object}}
\newcommand{\Value}{\set{Value}}
\newcommand{\FRM}[2]{\ensuremath{{#1} ~ \textit{frm} ~ {#2}}}
\newcommand{\REF}{\textit{Ref}}
\newcommand{\REFTYPE}{\textit{refT}}
\newcommand{\CName}{\textit{ClassName}}
\newcommand{\IName}{\textit{InterfaceName}}
\newcommand{\MName}{\textit{MethodName}}
\newcommand{\CT}[2]{\textit{CT}({#1}, {#2})}
\newcommand{\MT}[1]{\textit{MT}({#1})}
\newcommand{\NULL}{\textit{null}}
\newcommand{\FName}{\textit{FieldName}}
\newcommand{\FIELDS}{\textit{fields}}
\newcommand{\IFIELDS}{\textit{ifields}}
\newcommand{\DFIELDS}{\textit{dfields}}
\newcommand{\METHS}{\textit{meths}}
\newcommand{\SEMMETH}{\textit{SemMeth}}
\newcommand{\FORMALS}{\textit{formals}}
\newcommand{\SELFTYPE}{\textit{selftype}}
\newcommand{\EXTENDS}{\ensuremath{\triangleright}}
\newcommand{\EXLOCS}[2]{\textit{extLocs}( {#1} \EXTENDS {#2})}
\newcommand{\DMTYPE}[2]{\textit{dmtype}({#1}, {#2})}
\newcommand{\SUPERS}[1]{\ensuremath{\textit{supers}({#1})}}


% Statements
\newcommand{\M}[3]{\synbracket{\vdash^{\Gamma} {#1}}_{#2}{#3}}
\newcommand{\MP}[4]{\synbracket{{#1} \vdash {#2}}_{#3}{#4}}
\newcommand{\DM}[3]{{\mathcal{D}}\synbracket{\vdash^{\Gamma} {#1}}_{#2}{#3}}
\newcommand{\SM}[3]{{\mathcal{S}}\synbracket{\vdash^{\Gamma} {#1}}_{#2}{#3}}
\newcommand{\DCT}{{\mathcal{D}}\synbracket{CT}}

% Type
\newcommand{\TypJudg}[2]{{#1} \vdash {#2} }
\newcommand{\OT}[2]{\ensuremath{{#1}\!\!<\!\!{#2}\!\!>}}
%\newcommand{\OT}[2]{{#1}\!\!<\!\!{#2}\!\!>\!\!}
\newcommand{\TYPECB}{\textit{tyc}}
\newcommand{\OWNED}[2]{\ensuremath{\textit{owned}({#1}, {#2})}}
\newcommand{\OWNER}[2]{\ensuremath{\textit{owner}({#1}, {#2})}}
\newcommand{\OWNERS}[2]{\ensuremath{\textit{owners}({#1}, {#2})}}

\newcommand{\REP}{\KW{rep}}
\newcommand{\PEER}{\KW{peer}}
\newcommand{\LOST}{\KW{lost}}
\newcommand{\SELF}{\KW{self}}
\newcommand{\ANY}{\KW{any}}
\newcommand{\JOIN}[2]{\ensuremath{{#1}\sqcup{#2}}}

\newcommand{\InferX}[1]{\Phi \vdash^{\Gamma} #1}
\newcommand{\TYInferX}[3]{#1 \vdash^{#2} #3}
\newcommand{\BInfer}[3]{#1 \vdash_{#2}^{\Gamma} #3}

% Subeffect
\newcommand{\SUBEFF}[1]{\Phi \vdash^{\Gamma} #1 }


% Satisfaction
\newcommand{\PSAT}[4]{#1 \vDash_{#3}^{#2} #4}
\newcommand{\SAT}[2]{#1 \vDash_{\theta}^{\Gamma} #2}
\newcommand{\NSAT}[2]{#1 \not\vDash_{\theta}^{\Gamma} #2}

% semantic footprint function
\newcommand{\EFFS}[1]{\textit{efs}(\ensuremath{#1})}

\newcommand{\PROV}[2]{{#1}\vdash^{\Gamma} \ensuremath{#2}}
\newcommand{\TYPROV}[3]{{#1}\vdash^{#2} \ensuremath{#3}}
\newcommand{\DEFEQUIVR}[1]{\stackrel{{\rm #1}}{=}}


%\newcommand{\MV}{\texttt{MV}}
\newcommand{\FV}{\texttt{FV}}
%\newcommand{\HR}[1]{\textit{HR}(#1)}

\newcommand{\CODEX}[1]{\texttt{#1}}
\newcommand{\CODE}[1]{\texttt{#1}}
\newcommand{\CODEIT}[1]{\textit{#1}}
\newcommand{\FORALL}{\textit{for all}}
\newcommand{\EXISTS}{\textit{exists}}
\newcommand{\IMPLIES}{\textit{implies}}
\newcommand{\IFF}{\textit{iff}}

% type setting for semantics
\newcommand{\fun}[1]{\textit{#1}}
\newcommand{\set}[1]{\textit{#1}}
\newcommand{\SYNTAXCATEGORY}[1]{\nonterm{#1}}
\newcommand{\KW}[1]{\texttt{\textbf{#1}}}
\newcommand{\DOM}{\fun{dom}}
\newcommand{\CODOM}{\fun{codom}}
\newcommand{\EXTEND}{\textit{Extend}}
\newcommand{\DEFAULT}{\mbox{\textit{default}}}
%\newcommand{\VALALLOC}{\KW{alloc}\!\!\downarrow}
%\newcommand{\VALVAR}[1]{{#1}\!\!\downarrow}
\newcommand{\PFAULT}{\textit{p-fault}}


\newcommand{\REGRW}{\mbox{\textit{regRW}}}
\newcommand{\READR}{\mbox{\textit{readR}}}
\newcommand{\READVAR}{\mbox{\textit{readVar}}}
\newcommand{\WRITER}{\mbox{\textit{writeR}}}
\newcommand{\FRESHR}{\mbox{\textit{freshR}}}
\newcommand{\BND}[1]{\mbox{\textit{bnd}}(#1)}

\newcommand{\Vars}{\set{Vars}}
\newcommand{\ProgVar}{\set{ProgVar}}
\newcommand{\QVar}{\set{QVar}}
\newcommand{\FTPSUB}{\ensuremath{\sqsubseteq}}
\newcommand{\FTPEQUIV}{\ensuremath{\equiv}}


\newcommand{\OK}{\fun{ok}}

% syntactic categories
\newcommand{\Expr}{\SYNTAXCATEGORY{E}}
\newcommand{\Exprs}{\SYNTAXCATEGORY{Exprs}}
\newcommand{\Stmt}{\SYNTAXCATEGORY{S}}
\newcommand{\Id}{\SYNTAXCATEGORY{Id}}
\newcommand{\Ids}{\SYNTAXCATEGORY{Ids}}
\newcommand{\Ins}{\SYNTAXCATEGORY{ins}}
\newcommand{\RE}{\SYNTAXCATEGORY{R}}
\newcommand{\Assrt}{\SYNTAXCATEGORY{Assrt}}
\newcommand{\TypeJudg}{\SYNTAXCATEGORY{TypingJudgment}}
\newcommand{\ClassTable}{\DOMAIN{ClassTable}}


\newenvironment{nscenter}
 {\parskip=0pt\par\nopagebreak\centering}
 {\par\noindent\ignorespacesafterend}


% ----- listings

\lstdefinelanguage{Scala}%
{morekeywords={abstract,%
  case,catch,char,class,%
  def,else,extends,final,finally,for,%
  if,import,implicit,%
  match,module,%
  new,null,%
  object,override,%
  package,private,protected,public,%
  for,public,return,super,%
  this,throw,trait,try,type,%
  val,var,%
  with,while,%
  yield,%
  UNum, output, eval%
  },%
  sensitive,%
  commentstyle=\color{blue},
  moredelim=**[is][\color{red}]{<}{>},
  morecomment=[l]//,%
  morecomment=[s]{/*}{*/},%
  morestring=[b]",%
  morestring=[b]',%
  showstringspaces=false%
}[keywords,comments,strings]%
\def\inline{\lstinline[basicstyle=\small]}

\lstset{escapeinside={<<}{>>}}

