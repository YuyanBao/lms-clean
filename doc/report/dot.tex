\section{Language}
\begin{figure}[!htb]
\small
\centering
\begin{tabular}{l | l}
\begin{tabular}{r l l}
\nonterm{non-track} & ::= & non track \\
& 0 & \\
$\alpha$, $\beta$, $\gamma$ & ::= &  track \\
& $\emptyset$ &  emptyset \\
& \{\nonterm{x}\} & singleton set \\
& $\alpha_1 \cup \alpha_2$  & set union \\
& $\alpha_1 \backslash \alpha_2$ & set subtraction\\
\nonterm{d} & ::=  & Dependency \\
& \nonterm{non-track} 	& not track \\
& \nonterm{track} & track \\
\nonterm{T} & ::= & Type \\
& \nonterm{Int}$^{0}$ & Integer \\
& \nonterm{\&Int}$^{\alpha}$ & Reference type \\
& (\nonterm{T}$_1$ $\to$ \nonterm{T}$_2)^{\alpha}$ & type of functions \\
\nonterm{t} & ::= & Term \\
& \nonterm{c} & constant \\
& \nonterm{x} & variable \\
& \nonterm{alloc} & allocation \\
& \nonterm{inc}(\nonterm{x}) & mutation \\
& \KW{let} \nonterm{x} = \nonterm{t}$_1$ \KW{in} \nonterm{t}$_2$  & let binding \\
& $\lambda$\nonterm{x}.\nonterm{t} & abstraction \\
& \nonterm{t}$_1$\nonterm{t}$_2$ & application \\
\end{tabular} &
\begin{tabular}{r l l}
$\Gamma$ & :=  & Type context \\
& $\emptyset$ & Empty context \\
& $\Gamma$, \nonterm{x}:\nonterm{T} \\
\nonterm{l} & $\in \mathcal{N}$  & Address \\
\nonterm{H} & := & Value environment \\
& $\emptyset$ & empty env \\
& \nonterm{H}, \nonterm{x}:\nonterm{v}  \\
\nonterm{v} & ::= & Value \\
& \nonterm{c} & constant \\
& \nonterm{l} & address \\
& <\nonterm{H}, $\lambda$\nonterm{x}.\nonterm{t}> & closure \\
\nonterm{e} $\in$ \nonterm{E} & ::= & Evaluation context \\
& [] & \\
& \KW{let} \nonterm{x}=[] \KW{in} \nonterm{e} \\
& \KW{let} \nonterm{x} = \nonterm{v} \KW{in} \nonterm{e} \\ 
& \nonterm{e} t \\
& \nonterm{v} e
\end{tabular}
\end{tabular} %
\caption{The syntax and semantic definitions of the programming language.}
\label{fig:syntax}
\end{figure}

\begin{figure}[!htb]
\begin{mathpar}
\begin{array}{l l}
\inferrule{}
{\Gamma \vdash c : Int^{0}} & \mbox{(TCST)}
\\
\inferrule{\Gamma(x) = T}
{\Gamma \vdash x : T} & \mbox{(TVAR)}
\\
\inferrule{}
{\Gamma \vdash alloc : \&Int^\emptyset} & \mbox{(TREF)} 
\\
\inferrule{\Gamma \vdash x : \&Int^{\alpha}}
{\Gamma \vdash inc(x) : Int^{0}} & \mbox{(TINC)}
\\
\inferrule{\Gamma \vdash t_1: Int^{0}  \\ \Gamma, x:Int^{0} \vdash t_2 : T_2^{d}}
{\Gamma \vdash \KW{let} ~ x = t_1 \KW{in} ~ t_2 : T_2^{d}} & \mbox{(TLET1)}
\\
\inferrule{\Gamma \vdash t_1: T_1^{~\alpha}  \\ \Gamma, x:T_1^{~\alpha \cup \{x\}} \vdash t_2 : T_2^{0}}
{\Gamma \vdash \KW{let} ~ x = t_1 \KW{in} ~ t_2 : T_2^{0}} & \mbox{(TLET2)}
\\
\inferrule{\Gamma \vdash t_1: T_1^{~\alpha}  \\ \Gamma, x:T_1^{~\alpha \cup \{x\}} \vdash t_2 : T_2^{~\beta}}
{\Gamma \vdash \KW{let} ~ x = t_1 \KW{in} ~ t_2 : T_2^{~\beta \backslash \{x\}}} & \mbox{(TLET3)}
\\
%\inferrule{ \Gamma, x : T_1^{\{x\}} \vdash t : T_2^{\emptyset} \\ \gamma = MV(t) }
%{\Gamma \vdash \lambda x. t : (T_1^{\emptyset} \to T_2^{\emptyset})^{~\gamma}} & \mbox{(TABS1)}\\
%\mbox{where } \gamma \cap \{x\} = \emptyset \\
%\\
%\inferrule{ \Gamma, x : T_1^{\{x\}} \vdash t : T_2^{\{x\}} \\ \gamma = MV(t) }
%{\Gamma \vdash \lambda x. t : (T_1^{\{\emptyset\}} \to T_2^{~\{x\}})^{~\gamma}} & \mbox{(TABS2)} \\
%\mbox{where } \gamma \cap \{x\} = \emptyset \\
\inferrule{ \Gamma, x : T_1^{0} \vdash t : T_2^{d} \\ \gamma = MV(t) }
{\Gamma \vdash \lambda x. t : (T_1^{0} \to T_2^{d})^{~\gamma}} & \mbox{(TABS1)} \\
\mbox{where } \gamma \cap \{x\} = \emptyset \\
\inferrule{ \Gamma, x : T_1^{\alpha \cup \{x\}} \vdash t : T_2^{d} \\ \gamma = MV(t) }
{\Gamma \vdash \lambda x. t : (T_1^{\alpha} \to T_2^{d})^{~\gamma}} & \mbox{(TABS2)} \\
\mbox{where } \gamma \cap \{x\} = \emptyset \\
\\
\inferrule{ \Gamma \vdash t_1 : (T_1^{~\alpha} \to T_2^{~\beta})^{~\gamma} \\ \Gamma \vdash t_2 : T_1^{\delta} \\ \gamma \cap \delta = \alpha }
{\Gamma \vdash t_1t_2 : T_2^{~[\delta \backslash x]\beta} } & \mbox{(TAPP)}
\end{array}
\end{mathpar}
\caption{Types}
\end{figure}

\begin{figure}[!htb]
\begin{mathpar}
\begin{array}{r l l l}
MV(c) & = & \emptyset \\
MV(x) & = & \{x\} & \mbox{if } \Gamma(x) = \&Int^{~\alpha}\\
MV(alloc) & = & \emptyset \\
MV(inc(x)) & = & \{x\} \\
MV(\KW{let} ~ x = t_1 ~ \KW{in} ~ t_2) & = & MV(t_1) \cup MV(t_2) - \{x\} \\
MV(\lambda x. t) & = & MV(t) - \{x\} \\
MV(t_1t_2) & = & MV(t_1) \cup MV(t_2) 
\end{array}
\end{mathpar}
\caption{Definition of MV}
\end{figure}
\clearpage
\subsection{Let rules}
\begin{figure}[!htb]
\begin{mathpar}
\begin{array}{l l}
\inferrule{\Gamma \vdash t_1: Int^{0}  \\ \Gamma, x:Int^{0} \vdash t_2 : T_2}
{\Gamma \vdash \KW{let} ~ x = t_1 ~ \KW{in} ~ t_2 : T_2} & \mbox{(TLET1)}
\\
\inferrule{\Gamma \vdash t_1: \&Int^{~\alpha}  \\ \Gamma, x:T_1^{~\alpha \cup \{x\}} \vdash t_2 : Int^{0}}
{\Gamma \vdash \KW{let} ~ x = t_1 ~ \KW{in} ~ t_2 : Int^{0}} & \mbox{(TLET2)}
\\
\inferrule{\Gamma \vdash t_1: \&Int^{~\alpha}  \\ \Gamma, x:\&Int^{~\alpha \cup \{x\}} \vdash t_2 : \&Int^{~\beta}}
{\Gamma \vdash \KW{let} ~ x = t_1 ~ \KW{in} ~ t_2 : \&Int^{~\beta \backslash \{x\}}} & \mbox{(TLET3)}
\\
\inferrule{\vdash t_1 : (y: T_1 \to T_2)^{\alpha} \\ \Gamma, x: (y: T_1 \to T_2)^{\alpha} \vdash t_2 : T}
{\Gamma \vdash \KW{let} ~ x = t_1 ~ \KW{in} ~ t_2 : T} & \mbox{(TLET4)}

\end{array}
\end{mathpar}
\begin{lstlisting}[language=Scala,basicstyle=\footnotesize\ttfamily]
let x1 = 1 in 1           // TLET1

let x2 = 1 in alloc()     // TLET1

let x3 = alloc in 1       // TLET2

let x4 = alloc in x4      // TLET3

let x5 = alloc in
  let x6 = alloc in 
    x5                    //TLET3
    
let x7 = alloc in
  let x8 = inc(x7) in     // TLET1
     x8    
\end{lstlisting}
\caption{Program examples of applying the rule \emph{LET}}

\begin{mathpar}
\begin{array}{l l}
\inferrule{\Gamma \vdash 1 : Int^{0} \\ \Gamma, x1: Int^{0} \vdash 1 : Int^{0}}
{\Gamma \vdash \KW{let} ~ x1 = 1 ~ \KW{in} ~ 1 : Int^{0}} & \mbox{(TLET1)} 
\\
\inferrule{\Gamma \vdash 1 : Int^{0} \\ \Gamma, x2: Int^{0} \vdash alloc : \&Int^{\emptyset}}
{\Gamma \vdash \KW{let} ~ x2 = 1 ~ \KW{in} ~ alloc : \&Int^{\emptyset}} & \mbox{(TLET1)}
\\
\inferrule{\Gamma \vdash alloc : \&Int^{\emptyset} \\ \Gamma, x3: \&Int^{\{x3\}} \vdash 1 : Int^{0}}
{\Gamma \vdash \KW{let} ~ x3 = alloc ~ \KW{in} ~ 1 : Int^{0}} & \mbox{(TLET2)}
\\
\inferrule{\Gamma \vdash alloc : \&Int^{\emptyset} \\ \Gamma, x4: \&Int^{\{x4\}} \vdash x4 : \&Int^{\{x\}}}
{\Gamma \vdash \KW{let} ~ x4 = alloc ~ \KW{in} ~ x4 : \&Int^{\{x4\} \backslash \{x4\}}} & \mbox{(TLET3)}
\\
\inferrule{\Gamma \vdash alloc : \&Int^{\emptyset} \\ \Gamma, x6 : \&Int{\{x6\}} \vdash x5 : \&Int^{\{x5\}} }
{\Gamma \vdash \KW{let} ~ x6 = alloc ~ \KW{in} ~ x5 : \&Int^{\{x5\} \backslash \{x6\}}} & \mbox{(TLET3)}
\\
\inferrule{\Gamma \vdash inc(x7) : Int^{\emptyset} \\ \Gamma, x8: Int^{\emptyset} \vdash x8 : Int^{0}}
{\Gamma \vdash \KW{let} ~ x8 = inc(x7) ~ \KW{in} ~ x8 : Int^{0}} & \mbox{(TLET1)}
\\
\end{array}
\end{mathpar}
\caption{Examples of applying LET rules}
\end{figure}

\subsection{Abstraction rules}
\begin{enumerate}
\item $\gamma$ is a set of variables that may be aliased.
\item $\forall x \in \gamma, \Gamma(x) : \&Int^{\alpha}$ 
\end{enumerate}
\begin{figure}[!htb]
\begin{mathpar}
\begin{array}{l l}
\inferrule{ \Gamma, x : Int^{0} \vdash t : T \\ \gamma = MV(t) }
{\Gamma \vdash \lambda x. t : (Int^{0} \to T)^{~\gamma}} & \mbox{(TABS1)} \\
\inferrule{ \Gamma, x : \&Int^{\alpha \cup \{x\}} \vdash t : T \\ \gamma = MV(t) }
{\Gamma \vdash \lambda x. t : (\&Int^{\alpha} \to T)^{~\gamma}} & \mbox{(TABS2)} \\
\inferrule{ \Gamma, x : (y: T_1 \to T_2)^{\alpha} \vdash t : T_3 \\ \gamma = MV(t) }
{\Gamma \vdash \lambda x. t : ((T_1 \to T_2)^{\alpha} \to T_3)^{\gamma}} & \mbox{(TABS3)} \\
\end{array}
\end{mathpar}

\begin{lstlisting}[language=Scala,basicstyle=\footnotesize\ttfamily]
def f1(x: Int) = alloc()        // (Int$^{0}$ -> &Int$^{\emptyset}$)$^{\emptyset}$ (by TABS1)
def f2(x: &Int) = alloc()       // (&Int$^{\emptyset}$ -> &Int$^{\emptyset}$)$^{\emptyset}$ (by TABS2)
def f3(x: &Int) = x             // (&Int$^{\emptyset}$ -> &Int$^{\{x\}}$)$^{\emptyset}$ (by TABS2)

let y = alloc() in
  def f4(x: &Int) = y           // (&Int$^{\emptyset}$ -> &Int$^{y}$)$^{\{y\}}$ (by TABS2)
  
def f5(x: (&Int$^{\emptyset}$ -> Int$^{x}$)$^{0}$ = { 1 }  // f5: (&Int$^{\emptyset}$ -> Int$^{x}$)$^{0}$) -> Int$^{\emptyset}$
\end{lstlisting}
\caption{Program example of applying the rule \emph{TABS}}

\begin{mathpar}
\begin{array}{l l}
\inferrule{ \Gamma, x : Int^{0} \vdash t : \&Int^{\emptyset} \\  MV(alloc) = \emptyset}
{\Gamma \vdash \lambda x. alloc : (Int^{0} \to \&Int^{\emptyset})^{\emptyset}} & \mbox{(TABS1)}
\\
\inferrule{ \Gamma, x : \&Int^{\{x\}} \vdash alloc : \&Int^{\emptyset} \\ MV(alloc) = \emptyset }
{\Gamma \vdash \lambda x. alloc : (\&Int^{\emptyset} \to \&Int^{\emptyset})^{\emptyset}} & \mbox{(TABS2)}
\\
\inferrule{ \Gamma, x : \&Int^{\{x\}} \vdash x : \&Int^{\{x\}} \\ MV(x) - \{x\} = \emptyset }
{\Gamma \vdash \lambda x. x : (\&Int^{\emptyset} \to \&Int^{~\{x\}})^{~\emptyset}} & \mbox{(TABS2)}
\\
\inferrule{ \Gamma, x : \&Int^{\{x\}} \vdash y : \&Int^{\{x\}} \\ MV(y) - \{x\} = \{y\} }
{\Gamma \vdash \lambda x. y : (\&Int^{\emptyset} \to \&Int^{~\{y\}})^{y}} & \mbox{(TABS2)}
\\
\inferrule{ \Gamma, x : \&Int^{\{x\}} \vdash y : \&Int^{\{x\}} \\ MV(y) - \{x\} = \{y\} }
{\Gamma \vdash \lambda x. y : (\&Int^{\emptyset} \to \&Int^{~\{y\}})^{y}} & \mbox{(TABS2)}
\end{array}
\end{mathpar}
\caption{Example of appying the rule TABS.}
\end{figure}

\clearpage
\subsection{Application rules}
\begin{figure}[!htb]
\begin{mathpar}
\begin{array}{l l}
\inferrule{ \Gamma \vdash t_1 : (x:Int^{0} \to T)^{~\gamma} \\ \Gamma \vdash t_2 : Int^{0} }
{\Gamma \vdash t_1t_2 : T} & \mbox{(TAPP1)} 
\\
\inferrule{ \Gamma \vdash t_1 : (x:\&Int^{~\alpha} \to \&Int^{~\beta})^{~\gamma} \\ \Gamma \vdash t_2 : \&Int^{\delta} \\ \gamma \cap \delta = \alpha }
{\Gamma \vdash t_1t_2 : \&Int^{~[\delta / x]\beta} } & \mbox{(TAPP2)}
\end{array}
\end{mathpar}
\end{figure}

\begin{lstlisting}[language=Scala,basicstyle=\footnotesize\ttfamily]
def f1(x: Int) = alloc()        // f1 : (Int$^{0} \to$ &Int$^{\emptyset})^{\emptyset}$
f1(0)  // f(0):  &Int$^{\emptyset}$

let y = alloc() in
  def f2(x: Int) = y           // (Int$^{0}$ -> &Int$^{\{y\}}$)$^{\{y\}}$ 
  f2(0)                        // f2(0): &Int$^{\{y\}}$  
  
def f3(x: &Int) = alloc()       // (&Int$^{\emptyset}$ -> &Int$^{\emptyset}$)$^{\emptyset}$ 
let z = alloc() in
  f3(z)                         // f3(z): &Int$^{\emptyset}$
  
def f4(x: &Int) = x             // (&Int$^{\emptyset}$ -> &Int$^{\{x\}}$)$^{\emptyset}$ 
let t = alloc() in
  f4(t)                         // f4(t): &Int$^{\{t\}}$

\end{lstlisting}

\begin{mathpar}
\begin{array}{l l}
\inferrule{ \Gamma \vdash f1 : (x:Int^{0} \to \&Int^{\emptyset})^{\emptyset} \\  \Gamma \vdash 0 : Int^{0} \\ }
{\Gamma \vdash f1~ 0 : \&Int^{\emptyset}} & \mbox{(TAPP1)} \\
\inferrule{ \Gamma \vdash f2 : (x:Int^{0} -> \&Int^{\{y\}})^{\{y\}} \\  \Gamma \vdash 0 : Int^{0} \\ }
{\Gamma \vdash f2~ 0 : \&Int^{\{y\}}} & \mbox{(TAPP1)} \\
\inferrule{ \Gamma \vdash f3 : (x:\&Int^{\emptyset} -> \&Int^{\emptyset})^{\emptyset} \\  \Gamma \vdash z : \&Int^{\{z\}} \\ \emptyset \cap \{z\} = \emptyset  }
{\Gamma \vdash f3 ~z : \&Int^{[z/x]\emptyset}} & \mbox{(TAPP2)}

\end{array}
\end{mathpar}

\clearpage
\subsection{Inductive cases}
\begin{lstlisting}[language=Scala,basicstyle=\footnotesize\ttfamily]
let f1 = fun(x: Int => { alloc() } ) in  // f1: (Int$^{0}$ -> &Int$^{\emptyset}$)$^{\emptyset}$
  let f2 = f1 in
    f2(2)                                // &Int$^{\emptyset}$ 

\end{lstlisting}
\begin{figure}[!htb]
\begin{mathpar}
\begin{array}{l l}
\inferrule{\Gamma \vdash f1 : (x: Int^{0} \to \&Int^{\emptyset})^{\emptyset}  \\ \Gamma, f2: (x: Int^{0} \to \&Int^{\emptyset})^{\emptyset} \vdash f2(2) : Int^{0}}
{\Gamma \vdash \KW{let} ~ f2 = f1 ~ \KW{in} ~ f2(2) : Int^{0}  } & \mbox{(TLET???)}  \\

\end{array}
\end{mathpar}
\end{figure}

\begin{figure}
\begin{lstlisting}[language=Scala,basicstyle=\footnotesize\ttfamily]
let y = alloc() in
  let f3 = fun(x: &Int => { y } ) in      // (&Int$^{\emptyset}$ -> &Int$^{y}$)$^{\{y\}}$ (by TABS2)    
     let f4 = f3 in 
        f4(y)
\end{lstlisting}
\begin{mathpar}
\begin{array}{l l}
\inferrule*
{\Gamma \vdash f3 : (x: \&Int^{\emptyset} \to \&Int^{y})^{\{y\}} \\ \Gamma \vdash y : \&Int^{\{y\}}}
{\inferrule{ \Gamma, f3: (x: \&Int^{\emptyset} \to \&Int^{y})^{\{y\}} \vdash f4(y) :  \&Int^{\{y\}} \\ \{y\} \cap \{y\} = \{y\}}
{\Gamma \vdash \KW{let} ~ f4 = f3 ~ \KW{in} ~ f4(y) : Int^{0}}}
\end{array}
\end{mathpar}
\caption{Examples that get type errors}
\end{figure}
\clearpage

\begin{figure}[!htb]
\begin{mathpar}
\begin{array}{l l}
%\inferrule[(ECST)]{}
%{H ~ | ~ c \mapsto H ~ | ~ c }
%
%\inferrule[(EVAR)]{ x : v \in H }
%{H ~ | ~ x \mapsto H ~ | ~ v }
%
%\inferrule[(EREF)]{ fresh(l) }
%{H ~ | ~ alloc  \mapsto H ~ | ~ l }
%
%\inferrule[(ELET-VAR)]{}
%{H ~ | ~ \KW{let} ~ x = y ~ \KW{in} ~ t \mapsto H ~ | ~ [y / x]t }

\inferrule{}
{\KW{let} ~ x = y ~ \KW{in} ~ t \mapsto [y/x]t} & 
\mbox{(ELET-VAR)}
\\

\inferrule{}
{\KW{let} ~ x = \lambda z.t  ~ \KW{in} ~ E[x y] \mapsto \KW{let} ~ x = \lambda z.t \KW{in} ~ E[[y/z]t] } &
\mbox{(ELET-ABS)}
\\

\inferrule{}
{\KW{let} ~ x = (\KW{let} ~ y = t_1 \KW{in} ~ t_2) ~ \KW{in} ~ t_3 \mapsto \KW{let} ~ y = t_1 ~ \KW{in} ~ \KW{let} ~ x = t_2 ~ \KW{in} ~ t_3} &
\mbox{(ELET-LET)}
\\

\inferrule{}
{E[t_1] \mapsto E[t_2]} \quad \mbox{\textbf{if} } t_1 \mapsto t_2 &
\mbox{(ET)}
\end{array}
\end{mathpar}
\caption{Semantics (Reduction)}
\end{figure}

\begin{lemma}[Substitution] If $\Gamma, x : T^{~\alpha \cup \{x\}} \vdash t : T^{\beta}$ and $\Gamma \vdash [y / x]t : [y/x]T^{\beta}$.
\end{lemma}

\begin{theorem}[Soundness] If $\Gamma \vdash E[t]:T^{~\alpha}$, then either $t$ is an answer $(t = x)$ or there exists $\Gamma', E', t'$, such that $E[t] \to E'[t']$ and $\Gamma, \Gamma' \vdash E'[t']: T^{~\alpha}$.
\end{theorem}