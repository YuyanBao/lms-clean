\section{Approach}
We use fraction permissions to denote ownerhip. Each object has a field called \emph{ownership} denoting by a floating number between 0 and 1 inclusive. The permission 0 means that the variable is not accessible anymore. The permission 1 means that the variable is mutable. If \CODE{o} is greater than 0 and less than 1, then it is readable.

There is a partial function \CODE{globalOwnership} that maps from variables to their permissions.
There is a partial function \CODE{globalAliasing} that maps from variables to set of variables. 

\begin{lstlisting}[language=Scala,basicstyle=\footnotesize\ttfamily]
val x = new mut Cell  [x -> 1][]
val y = &x            [x -> 0.5, y -> 0.5][x -> {y}] 
val z = &x            [x - 0.25, z - 0.25, y - 0.5][x - {y,z}]
x.f = 5               [x -> 1, z -> 0, y -> 0][]  

\end{lstlisting} 

\begin{lstlisting}[language=Scala,basicstyle=\footnotesize\ttfamily]
if (b){
val x = new mut Cell  [x -> 1][]
val y = &x            [x -> 0.5, y -> 0.5][x -> {y}] 
val z = &x            [x - 0.25, z - 0.25, y - 0.5][x - {y, z}]
x.f = 5               [x -> 1, z -> 0, y -> 0][]  
}
val t = x             [x -> 0, z -> 0, y -> 0][]  
\end{lstlisting} 

\begin{lstlisting}[language=Scala,basicstyle=\footnotesize\ttfamily]
val t = if (b){
val x = new mut Cell  [x -> 1][]
val y = &x            [x -> 0.5, y -> 0.5][x -> {y}] 
val z = &x            [x - 0.25, z - 0.25, y - 0.5][x ->{y, z}]
x.f = 5 
x                     [x -> 1, z -> 0, y -> 0][]  
}                     [t -> 1, x -> 0, z -> 0, y -> 0][]  
\end{lstlisting} 

\subsection{Methods}
There is a partial function \CODE{globalFun} that maps from method names to pairs of parameters' permissions and return value's permission.
\begin{lstlisting}[language=Scala,basicstyle=\footnotesize\ttfamily]
[f1-> ([1], -1)] 
def f1(c: Cell) : Unit = {    [c -> 1][] // assumption
  val c0 = c;                 [c-> 0; c0 -> 1][]
}    
def client() = {
  val cc = new mut Cell;      [cc -> 1][] 
  f1(cc)                      [cc -> 0][] // assert cc -> 1
  cc.f                        // error
}
\end{lstlisting}

\begin{lstlisting}[language=Scala,basicstyle=\footnotesize\ttfamily]
[f2-> ([1], 1)] 
def f2(c: Cell) : Cell = {    [c -> 1][]  // assumption
  val c0 = c;                 [c-> 0; c0 -> 1][]
  c0.f = 5;                   [c-> 0; c0 -> 1][]
  c0                                     // assert c0 -> 1
}    
def client() = {
  val cc = new mut Cell;      [cc -> 1][]
  val ct = f1(cc)             [cc -> 0, ct -> 1][] // assert cc == 1; 
  ct.f = 5;                   // ok   
}
\end{lstlisting}

\begin{lstlisting}[language=Scala,basicstyle=\footnotesize\ttfamily]
[f3-> ([0.5], -1)] 
def f3(c: & Cell) : Unit = {  [c -> 0.5][] // assumption
  val c0 = c;                 [c -> 0.25, c0 -> 0.25][c -> {c0}]
  c.f = 5;   // error         
}
def client() = {
  val cc = new mut Cell;      [cc -> 1][]
  f3(&cc)                     [cc -> 1][] // assert cc > 0
  cc.f = 5;                   [cc -> 1][] // ok
}  
\end{lstlisting}

\begin{lstlisting}[language=Scala,basicstyle=\footnotesize\ttfamily]
[f4-> ([0.5], 1)] 
def f4(c: & Cell) : Unit = {  [c -> 0.5][] // assumption
  val t = c.f
  val c0 = new mut Cell(t);   [c -> 0.5, c0 -> 1][]
  c0                                      // assert c0 -> 1 
}
def client() = {
  val cc = new mut Cell;      [cc -> 1][]
  val ct = f4(&cc)            [cc -> 1, ct -> 1] // assert cc > 0
  ct.f = 5;                   // ok
}  
\end{lstlisting}

The following two cases may not be sensible. It needs to track aliasing in the method body.
\begin{lstlisting}[language=Scala,basicstyle=\footnotesize\ttfamily]
[f5-> ([0.5], 0.5)] 
def f5(c: & Cell) : & Cell = {  [c -> 0.5][] // assumption
  val cc = new mut Cell;
  &cc                                      // assert cc > 0
}
def client() = {
  val cc = new mut Cell;      [cc -> 1][]
  val ct = f5(&cc)            [cc -> 1, ct -> 0.5][?]  // assert cc > 0 
  ct.f = 5;                    // error
  cc.f = 5;                   [cc -> 1, ct -> ?][?]
}  
\end{lstlisting}

\begin{lstlisting}[language=Scala,basicstyle=\footnotesize\ttfamily]
[f6-> ([1], 0.5)] 
def f6(c: Cell) : & Cell = {  [c -> 1][] // assumption
  &c                                      // assert cc > 0
}
def client() = {
  val cc = new mut Cell;      [cc -> 1][]
  val ct = f6(cc)             [cc -> 0, ct -> 0.5][?]  // assert cc == 1
  ct.f = 5;                   // error
  cc.f = 5;                   // error
}  
\end{lstlisting}

\clearpage
\subsection{Higher-order functions}
%\CODE{f} cannot escape from \CODE{withFile}. So \CODE{fn} has to be \emph{local}:
%\begin{itemize}
%\item \CODE{f} cannot be returned from \CODE{fn}
%\item the return value of \CODE{fn} is readonly.
%\item \CODE{fn} can only be invoked by \emph{local} functions.
%\end{itemize}
\begin{itemize}
\item There is a partial function \CODE{globalOwnership} that maps from variables to their permissions.
\item There is a partial function \CODE{globalAliasing} that maps from variables to set of variables. 
\item There is a partial function \CODE{globalFun} that maps from method names to a tuple of parameters' permissions,  return value's permission and aliasing information.
\end{itemize}

\paragraph{1st class values}

The following example shows that the method body is not type checked. 
\begin{lstlisting}[language=Scala,basicstyle=\footnotesize\ttfamily]
[withCell -> ([-1], 0.5, [?]), fn -> ([0.5], 0.5, [p1]), client -> ([-1], -1, [])]
def withCell(n: Int)(fn: & Cell =>  & Cell): & Cell = {
  val c = new Cell(n);      [c -> 1][]
  try val ret = fn(& c)     [c -> 0.5, ret -> 0.5][c -> {ret}]  
  finally c.free            [c -> 0, ret -> 0][]
  ret                       [c -> 0, ret -> 0][]    // error
}
def client(){
  val c1: Cell = withCell(5){ c => c};
}
\end{lstlisting}

Question marks mean that the return value is aliasing with somebody, but is definitely not with parameters.
\begin{lstlisting}[language=Scala,basicstyle=\footnotesize\ttfamily]
[withCell -> ([-1], 0.5, [?]), fn -> ([0.5], 0.5, [?]), client -> ([-1], -1, [])]
def withCell(n: Int)(fn: & Cell =>  & Cell): & Cell = {
  val c = new Cell(n);      [c -> 1][]
  try val ret = fn(& c)     [c -> 0.5, ret -> 0.5][]  
  finally c.free            [c -> 0, ret -> 0.5][] // treated as mutation
  ret                       [c -> 0, ret -> 0.5]    // ok
}
def client(){
  val c1 : Cell = withCell(5){ c => val t = new Cell(c.f); &t }.
}
\end{lstlisting}

%\begin{lstlisting}[language=Scala,basicstyle=\footnotesize\ttfamily]
%[withCell -> ([-1], 0.5, []), fn -> ([1], 1, [p1]), client -> ([-1], -1, [])]
%def withCell(n: Int)(fn:  Cell =>  Object): Object = {
%  val c = new Cell(n);      [c -> 1][]
%  try val ret = fn(c)       [c -> 0, ret -> 1][]  
%  finally c.free            [c -> 0, ret -> 1][]  // error
%  ret                       [c -> 0, ret -> 1][]
%}
%def client(){
%  val c1: Cell = withCell(5){ f => f};
%}
%\end{lstlisting}

\emph{Summary:} 
\begin{itemize}
\item The apporach is \emph{not modular} because without knowing the implementation of \CODE{fn}, the aliasing information of \CODE{withCell} cannot be known.
\item Function parameters are like interfaces with one function in Java.
\end{itemize}


\paragraph{2nd class values} 

%The following example shows that the method body is not type-checked by \CODE{@local}. 
%\begin{lstlisting}[language=Scala,basicstyle=\footnotesize\ttfamily]
%[withCell -> ([-1], 0.5, []), fn -> ([0.5], 0.5, [p1])]
%def withCell(n: Int)(fn: & Cell =>  & Cell): & Cell = {
%  @local val c = new Cell(n);    [c -> 1][]
%  try val ret = fn(& c)          [c -> 0.5, ret -> 0.5][]  // not type-checked by @local
%  finally c.free                 [c -> 1, ret -> 0][]  
%  ret                            [c -> 0, ret -> 0][]     
%}
%def client(){
%  val c1: Cell = withCell(5){ f => f};
%}
%\end{lstlisting}

\[
\]

The signature of \CODE{fn} means that the return value of \CODE{fn} is not aliased with the parameter. 

Question marks mean that the return value is aliasing with somebody, but is definitely not with parameters.
\begin{lstlisting}[language=Scala,basicstyle=\footnotesize\ttfamily]
[withCell -> ([-1], 0.5, []), fn -> ([0.5], 0.5, [?])]
def withCell(n: Int)(fn: @local & Cell =>  & Cell ): & Cell = {
  @local val c = new Cell(n);    [c -> 1][]
  try val ret = fn(& c)          [c -> 0.5, ret -> 0.5][]  
  finally c.free                 [c -> 0, ret -> 0.5][]  
  ret                            [c -> 0, ret -> 0.5][]     
}
def client(){
  val c1: Cell = withCell(5){ 
        c => c      // error
  }
  // first-class values are also second-class.
  val c2: Cell = withCell(5){ c => val t = new Cell(c.f); & t}; [c2 -> 0.5][]
  // don't know who c2 is aliased with.
}
\end{lstlisting}



%\begin{lstlisting}[language=Scala,basicstyle=\footnotesize\ttfamily]
%[withCell -> ([-1], 0.5, []), fn -> ([0.5], 0.5, [?])]
%def withCell(n: Int)(fn: & Cell =>  Cell): Cell = {
%  @local val c = new Cell(n);    [c -> 1][]
%  try val ret = fn(& c)          [c -> 0.5, ret -> 0.5][]  // ret is 2nd-class value
%  finally c.free                 [c -> 0, ret -> 0.5][]  
%  ret                            [c -> 0, ret -> 0.5][]     
%}
%def client(){
%  val c1: Cell = withCell(5){ c => val t = new Cell(c.f); & t}; [c1 -> 0.5][]
%  // don't know who c1 is aliased with.
%}
%\end{lstlisting}

Parameters and return values are both @local.
\begin{lstlisting}[language=Scala,basicstyle=\footnotesize\ttfamily]
[withCell -> ([-1], -1, []), fn -> ([0.5], 0.5, [p1])]
def withCell(n: Int)(fn: @local & Cell =>  @local & Cell): Unit = {
  @local val c = new Cell(n);    [c -> 1][]
  try val ret = fn(& c)          [c -> 0.5, ret -> 0.5][c -> {ret}]
  ret.f = 5                      [c -> 0, ret -> 1][]
  finally c.free                 [c -> 0, ret -> 0][]  
}
def client(){
  withCell(5){ c => c }; [][]
}
\end{lstlisting}

Nested @local: need to disassemble \CODE{fn}. Is it allowed by @local type system?
\begin{lstlisting}[language=Scala,basicstyle=\footnotesize\ttfamily]
[withCell -> ([-1], -1, []), fn -> ([0.5], -1, [p1]), <fnn> -> ([0.5], 0.5, [p1])]
def withCell(n: Int)
            (fn: @local & Cell =>  (@local & Cell => @local & Cell)): Unit = {
  @local val c = new Cell(n)     [c -> 1][]
  try{ 
    val <ret_fn> = fn(& c)         [c -> 0.5][]  
    val ret = ret_fn(c)          [c -> 0.5, ret -> 0.5][c -> {ret}]  
    ret.f = 5                    [c -> 0, ret -> 1][]
  }
  finally ret.free               [c -> 0, ret -> 0][]  
}
def client(){
  withCell(5){ c => (c => c) }; [][]
}
\end{lstlisting}

%Other cases:
%\begin{lstlisting}[language=Scala,basicstyle=\footnotesize\ttfamily]
%def withCell(n: Int)(fn: @local Cell =>  Unit): Unit = {
%  @local val c = new Cell(n);    
%  fn(c)                        
%}
%def client(){
%  @local var c0 = null        
%  withCell(5){ c => c0 = c) } 
%  free c0                     
%}
%\end{lstlisting}

\clearpage

\subsection{Structs}

\cite{haller2010capabilities}
We use @rep to denote the object is owned by the enclosing structure.
We use @sep to denote the memory regions reachable from an object is disjoint with other objects.

\begin{lstlisting}[language=Scala,basicstyle=\footnotesize\ttfamily]
@sep              //
struct Pair{
  val x : Cell
  val y : Cell
}

[shift -> ([1, -1], -1, [])]
def shift(@ref @mut c: Cell, i: Int) : Unit { // c is not used after the call to shift
   c.f = c.f + i      
}

[shift -> ([1, -1], 1, [])]
def shift1(@mut c: Cell, i: Int) : @mut Cell { // @mut reference has to be returned
   c.f = c.f + i
   c      
}

[incr -> ([0.5, -1], -1, [])]
def incr(@ref c: Cell, i: Int) : Int {       // c is not used after the call to shift
   c.f + i        
}

[incr -> ([0.5, -1], -1, [])]
def incr1(@ref c: Cell, i: Int) : @ref Cell {  <not safe>
   val ret = new Cell(c.f + i)
   @ref ret
}

[withPair -> ([1, 1], -1, []), fn -> ([0.5], -1, [])]
def withPair(@sep c1: Cell, @sep c2: Cell)(fn: @local @sep @ref Pair => Int) : Int {
 @local var p = new Pair(c1, c2)  [p -> 1][]
 fn(@ref p)                       [p -> 1][]                     
 shift(@ref @mut p.x)             [p.x -> 1, p.y -> 1][]
 shift1(@mut p.x)                 [p.x -> 1, p.y -> 1][]
 val t = incr(@ref p.x)           [p.x -> 1, p.y -> 1][]
 free(p)                          [p -> 0][]  // assert(p.x -> 1, p.y -> 1)
}

def client1(){
  val c = new Cell(1)           [c -> 1][] 
  val c1 = & c                  [c -> 0.5][c1 -> 0.5][c -> {c1}]
  val p = withPair(c, c1){ ... }      // error   
}

def client2(){
  val c = new Cell(1)                         [c -> 1][] 
  val c1 = new Cell(2)                        [c -> 1, c1 -> 1][] 
  val sum = withPair(c, c1){ p => p.x + p.y } [c -> 0, c1 -> 0][]       
}

\end{lstlisting}

\emph{Array}
\begin{lstlisting}[language=Scala,basicstyle=\footnotesize\ttfamily]
[add -> ([1, -1], 1, []), withArray -> ([1], 1, []), fn -> ([0.5], -1, [])]
def add(arr: Array[@sep Pair], idx : Int) : Array[@sep Pair]{
  val p = new Pair(new Cell(1), new Cell(2))   [arr -> 1, p -> 1]
  arr[idx] = p                                 [arr#0 -> 1, $\cdots$, arr#arr.length - 1 -> 1] 
  arr               // assert(arr#0 -> 1, $\cdots$, arr#arr.length - 1 -> 1)
}
def withArray(arr: Array[@sep Pair])(fn: @sep & Pair => Int) : Array[@sep Pair]{
                           [arr -> 1]
  arr.map(x => fn(x))      [arr#0 -> 1, $\cdots$, arr#arr.length - 1 -> 1]          
  arr               // assert(arr#0 -> 1, $\cdots$, arr#arr.length - 1 -> 1)
}

def client(){
  val arr = Array[@sep Pair](5)                    [arr -> 1][]
  val arr1 = add(arr, 0)                           [arr-> 0, arr1 -> 1][]
  val arr2 = withArray(arr1){p => p.x + p.y } [arr-> 0, arr1 -> 0, arr2 -> 1][]
}
\end{lstlisting}

\emph{HashMap}
\begin{lstlisting}[language=Scala,basicstyle=\footnotesize\ttfamily]
def client(){
  val m = mutable.HashMap[Int, @sep Pair]()
  val imm = Hash[Int, @sep Pari]()
}
\end{lstlisting}

\clearpage
\subsection{Properties}
\begin{itemize}
\item At any given time, you can have either one mutable reference or any number of immutable references.
\item References must always be valid.
\end{itemize}

\subsection{Questions}
%need a way to distinguish mutable variable and mutable heap locations
%
%\begin{lstlisting}[language=Scala]
%val x = new mut Cell // ok, the instance of Cell is mutable
%val x = new Cell     // ok, the instance of Cell is not mutable
%var y = new mut Cell // not ok?
%var y = new Cell     // 
%
%\end{lstlisting}

%The \CODE{typeMap} does not have $\&$ information.
%
%I don't see the $\lambda$ node.







% Local Variables:
% word-wrap: 1
% auto-fill-function: nil
% End:
 


